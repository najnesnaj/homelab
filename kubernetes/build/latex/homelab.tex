%% Generated by Sphinx.
\def\sphinxdocclass{report}
\documentclass[letterpaper,10pt,english]{sphinxmanual}
\ifdefined\pdfpxdimen
   \let\sphinxpxdimen\pdfpxdimen\else\newdimen\sphinxpxdimen
\fi \sphinxpxdimen=.75bp\relax
\ifdefined\pdfimageresolution
    \pdfimageresolution= \numexpr \dimexpr1in\relax/\sphinxpxdimen\relax
\fi
%% let collapsible pdf bookmarks panel have high depth per default
\PassOptionsToPackage{bookmarksdepth=5}{hyperref}

\PassOptionsToPackage{booktabs}{sphinx}
\PassOptionsToPackage{colorrows}{sphinx}

\PassOptionsToPackage{warn}{textcomp}
\usepackage[utf8]{inputenc}
\ifdefined\DeclareUnicodeCharacter
% support both utf8 and utf8x syntaxes
  \ifdefined\DeclareUnicodeCharacterAsOptional
    \def\sphinxDUC#1{\DeclareUnicodeCharacter{"#1}}
  \else
    \let\sphinxDUC\DeclareUnicodeCharacter
  \fi
  \sphinxDUC{00A0}{\nobreakspace}
  \sphinxDUC{2500}{\sphinxunichar{2500}}
  \sphinxDUC{2502}{\sphinxunichar{2502}}
  \sphinxDUC{2514}{\sphinxunichar{2514}}
  \sphinxDUC{251C}{\sphinxunichar{251C}}
  \sphinxDUC{2572}{\textbackslash}
\fi
\usepackage{cmap}
\usepackage[T1]{fontenc}
\usepackage{amsmath,amssymb,amstext}
\usepackage{babel}



\usepackage{tgtermes}
\usepackage{tgheros}
\renewcommand{\ttdefault}{txtt}



\usepackage[Bjarne]{fncychap}
\usepackage{sphinx}

\fvset{fontsize=auto}
\usepackage{geometry}


% Include hyperref last.
\usepackage{hyperref}
% Fix anchor placement for figures with captions.
\usepackage{hypcap}% it must be loaded after hyperref.
% Set up styles of URL: it should be placed after hyperref.
\urlstyle{same}

\addto\captionsenglish{\renewcommand{\contentsname}{Contents:}}

\usepackage{sphinxmessages}
\setcounter{tocdepth}{1}



\title{Homelab}
\date{May 31, 2025}
\release{1}
\author{Jan Jansen}
\newcommand{\sphinxlogo}{\vbox{}}
\renewcommand{\releasename}{Release}
\makeindex
\begin{document}

\ifdefined\shorthandoff
  \ifnum\catcode`\=\string=\active\shorthandoff{=}\fi
  \ifnum\catcode`\"=\active\shorthandoff{"}\fi
\fi

\pagestyle{empty}
\sphinxmaketitle
\pagestyle{plain}
\sphinxtableofcontents
\pagestyle{normal}
\phantomsection\label{\detokenize{index::doc}}


\sphinxstepscope


\chapter{intro}
\label{\detokenize{intro:intro}}\label{\detokenize{intro::doc}}

\section{Homelab}
\label{\detokenize{intro:homelab}}
\sphinxAtStartPar
A homelab is an environment where I can try things out, without upsetting a CISO, CDO, CITO ….
The system is an old HP proliant gen8, with 2TB of space and 256GB of memory, and bi\sphinxhyphen{}XEON with 16 cores in total.


\section{Virtualisation}
\label{\detokenize{intro:virtualisation}}
\sphinxAtStartPar
I choose proxmox as a virtualisation platform, it runs linux containers (LXC), which are kind of cool since they are created quickly, launched quickly. Some problems arise since there are still shared resources with the host…

\sphinxAtStartPar
Proxmox supports virtual machines (VM) as well.

\sphinxAtStartPar
VM are used as kubernetes nodes, since using LXC was problematic.
This setup allows to run a complete kubernetes multiple node cluster on a single machine.


\section{Containerisation}
\label{\detokenize{intro:containerisation}}
\sphinxAtStartPar
Containerization, in simple terms, is like packing a lunchbox for software. It’s a way to bundle an application with everything it needs to run—code, libraries, settings, and dependencies—into a single, portable package called a container. This makes it easy to move and run the software consistently on any computer, server, or cloud, without worrying about differences in the environment.


\section{Kubernetes}
\label{\detokenize{intro:kubernetes}}\begin{itemize}
\item {} 
\sphinxAtStartPar
Organizes Containers: It decides which computers (servers) should run each container, spreading them out to avoid overloading any one server. It’s like assigning chefs to different kitchen stations to keep things efficient.

\item {} 
\sphinxAtStartPar
Scales Automatically: If your app gets super popular (like a sudden rush of customers), Kubernetes can add more containers to handle the demand. If things slow down, it removes extras to save resources.

\item {} 
\sphinxAtStartPar
Keeps Things Running: If a container crashes (like a lunchbox spilling), Kubernetes quickly replaces it with a new one so your app stays available.

\item {} 
\sphinxAtStartPar
Manages Traffic: It directs users to the right containers, ensuring everyone gets access to the app without delays, like guiding customers to the right food station.

\end{itemize}

\sphinxAtStartPar
Updates Smoothly

\sphinxstepscope


\chapter{Kubernetes Architecture  MetalLB, Traefik}
\label{\detokenize{setup:kubernetes-architecture-metallb-traefik}}\label{\detokenize{setup:kubernetes-architecture}}\label{\detokenize{setup::doc}}
\sphinxAtStartPar
To make \sphinxcode{\sphinxupquote{www.example.com}} accessible from my laptop.


\section{Overview of  Setup}
\label{\detokenize{setup:overview-of-setup}}\begin{quote}

\sphinxAtStartPar
running a Kubernetes cluster on Talos Linux nodes, hosted on a Proxmox virtualization platform. The setup includes:
\end{quote}
\begin{itemize}
\item {} 
\sphinxAtStartPar
\sphinxstylestrong{MetalLB} for load balancing

\item {} 
\sphinxAtStartPar
\sphinxstylestrong{Traefik} as an ingress controller

\item {} 
\sphinxAtStartPar
An \sphinxstylestrong{Ingress} resource to route traffic to a service for \sphinxcode{\sphinxupquote{www.example.com}}

\item {} 
\sphinxAtStartPar
\sphinxstylestrong{laptop} resolves \sphinxcode{\sphinxupquote{www.example.com}} to \sphinxcode{\sphinxupquote{10.10.10.10}} via \sphinxcode{\sphinxupquote{/etc/hosts}}

\item {} 
\sphinxAtStartPar
A \sphinxstylestrong{Proxmox LXC container} acts as a router/DHCP server (\sphinxcode{\sphinxupquote{192.168.1.200}}) on a network bridge (\sphinxcode{\sphinxupquote{vmbr1}})

\item {} 
\sphinxAtStartPar
a router assigns IPs to three Talos nodes and one Talos client

\end{itemize}

\sphinxAtStartPar
Below, each part is explained in detail.


\section{1. Proxmox and Networking Setup}
\label{\detokenize{setup:proxmox-and-networking-setup}}
\sphinxAtStartPar
Proxmox is the virtualization platform hosting your infrastructure, including the Talos VMs and the router container.

\sphinxAtStartPar
\sphinxstylestrong{Proxmox Network Bridge (vmbr1)}
\begin{itemize}
\item {} 
\sphinxAtStartPar
Acts as a virtual switch that connects VMs and containers.

\item {} 
\sphinxAtStartPar
Configured to handle networking for Talos nodes and the client.

\item {} 
\sphinxAtStartPar
Devices on this bridge typically reside in the \sphinxcode{\sphinxupquote{192.168.1.0/24}} subnet.

\end{itemize}

\sphinxAtStartPar
\sphinxstylestrong{LXC Container (Router/DHCP Server at 192.168.1.200)}
\begin{itemize}
\item {} 
\sphinxAtStartPar
Provides dynamic IP assignment via DHCP.

\item {} 
\sphinxAtStartPar
Routes traffic between the \sphinxcode{\sphinxupquote{vmbr1}} subnet and external networks.

\item {} 
\sphinxAtStartPar
May also offer NAT, DNS, or firewall services.

\end{itemize}

\sphinxAtStartPar
\sphinxstylestrong{Network Flow}
\begin{itemize}
\item {} 
\sphinxAtStartPar
Talos nodes and the client receive IPs (e.g., \sphinxcode{\sphinxupquote{192.168.1.201\sphinxhyphen{}204}}) from this container.

\item {} 
\sphinxAtStartPar
The container enables traffic to flow between your laptop’s network and the Kubernetes cluster.

\end{itemize}


\section{2. Talos Linux Kubernetes Cluster}
\label{\detokenize{setup:talos-linux-kubernetes-cluster}}
\sphinxAtStartPar
\sphinxstylestrong{Three Talos Cluster Nodes}
\begin{itemize}
\item {} 
\sphinxAtStartPar
VMs on Proxmox configured as Kubernetes nodes.

\item {} 
\sphinxAtStartPar
a control plane 2 worker nodes

\item {} 
\sphinxAtStartPar
IPs assigned by DHCP from the LXC router.

\end{itemize}

\sphinxAtStartPar
\sphinxstylestrong{One Talos Client}
\begin{itemize}
\item {} 
\sphinxAtStartPar
a LXC device used for managing the cluster using \sphinxcode{\sphinxupquote{talosctl}}.

\item {} 
\sphinxAtStartPar
Communicates with the control plane via the Talos API.

\end{itemize}

\sphinxAtStartPar
\sphinxstylestrong{Talos Networking}
\begin{itemize}
\item {} 
\sphinxAtStartPar
Nodes are assigned IPs via \sphinxcode{\sphinxupquote{vmbr1}}.

\item {} 
\sphinxAtStartPar
Kubernetes networking handles intra\sphinxhyphen{}cluster communication.

\end{itemize}


\section{3. Kubernetes Networking with MetalLB and Traefik}
\label{\detokenize{setup:kubernetes-networking-with-metallb-and-traefik}}
\sphinxAtStartPar
\sphinxstylestrong{MetalLB}
\begin{itemize}
\item {} 
\sphinxAtStartPar
Provides load balancing by assigning external IPs (e.g., \sphinxcode{\sphinxupquote{10.10.10.10}}) to LoadBalancer services.

\item {} 
\sphinxAtStartPar
Operates in Layer 2 (ARP) or BGP mode.

\item {} 
\sphinxAtStartPar
A node “owns” the external IP and routes traffic to the appropriate pod.

\end{itemize}

\sphinxAtStartPar
\sphinxstylestrong{How MetalLB Works}
\begin{itemize}
\item {} 
\sphinxAtStartPar
Configured with a pool like \sphinxcode{\sphinxupquote{10.10.10.0/24}}.

\item {} 
\sphinxAtStartPar
Ensures external traffic is routed to the correct service or pod via ARP.

\end{itemize}

\sphinxAtStartPar
\sphinxstylestrong{Traefik as Ingress Controller}
\begin{itemize}
\item {} 
\sphinxAtStartPar
A reverse proxy and ingress controller running in the cluster.

\item {} 
\sphinxAtStartPar
Exposed via a LoadBalancer service assigned the IP \sphinxcode{\sphinxupquote{10.10.10.10}}.

\item {} 
\sphinxAtStartPar
Handles HTTP/S requests and routes them according to Ingress rules.

\end{itemize}

\sphinxAtStartPar
\sphinxstylestrong{Ingress Resource Example}

\begin{sphinxVerbatim}[commandchars=\\\{\}]
\PYG{n+nt}{apiVersion}\PYG{p}{:}\PYG{+w}{ }\PYG{l+lScalar+lScalarPlain}{networking.k8s.io/v1}
\PYG{n+nt}{kind}\PYG{p}{:}\PYG{+w}{ }\PYG{l+lScalar+lScalarPlain}{Ingress}
\PYG{n+nt}{metadata}\PYG{p}{:}
\PYG{+w}{  }\PYG{n+nt}{name}\PYG{p}{:}\PYG{+w}{ }\PYG{l+lScalar+lScalarPlain}{example\PYGZhy{}ingress}
\PYG{+w}{  }\PYG{n+nt}{annotations}\PYG{p}{:}
\PYG{+w}{    }\PYG{n+nt}{traefik.ingress.kubernetes.io/router.entrypoints}\PYG{p}{:}\PYG{+w}{ }\PYG{l+lScalar+lScalarPlain}{web,}\PYG{l+lScalar+lScalarPlain}{ }\PYG{l+lScalar+lScalarPlain}{websecure}
\PYG{n+nt}{spec}\PYG{p}{:}
\PYG{+w}{  }\PYG{n+nt}{rules}\PYG{p}{:}
\PYG{+w}{  }\PYG{p+pIndicator}{\PYGZhy{}}\PYG{+w}{ }\PYG{n+nt}{host}\PYG{p}{:}\PYG{+w}{ }\PYG{l+lScalar+lScalarPlain}{www.example.com}
\PYG{+w}{    }\PYG{n+nt}{http}\PYG{p}{:}
\PYG{+w}{      }\PYG{n+nt}{paths}\PYG{p}{:}
\PYG{+w}{      }\PYG{p+pIndicator}{\PYGZhy{}}\PYG{+w}{ }\PYG{n+nt}{path}\PYG{p}{:}\PYG{+w}{ }\PYG{l+lScalar+lScalarPlain}{/}
\PYG{+w}{        }\PYG{n+nt}{pathType}\PYG{p}{:}\PYG{+w}{ }\PYG{l+lScalar+lScalarPlain}{Prefix}
\PYG{+w}{        }\PYG{n+nt}{backend}\PYG{p}{:}
\PYG{+w}{          }\PYG{n+nt}{service}\PYG{p}{:}
\PYG{+w}{            }\PYG{n+nt}{name}\PYG{p}{:}\PYG{+w}{ }\PYG{l+lScalar+lScalarPlain}{example\PYGZhy{}service}
\PYG{+w}{            }\PYG{n+nt}{port}\PYG{p}{:}
\PYG{+w}{              }\PYG{n+nt}{number}\PYG{p}{:}\PYG{+w}{ }\PYG{l+lScalar+lScalarPlain}{80}
\end{sphinxVerbatim}

\sphinxAtStartPar
This configuration tells Traefik to forward traffic for \sphinxcode{\sphinxupquote{www.example.com}} to the \sphinxcode{\sphinxupquote{example\sphinxhyphen{}service}} on port \sphinxcode{\sphinxupquote{80}}.


\section{4. Laptop Configuration}
\label{\detokenize{setup:laptop-configuration}}
\sphinxAtStartPar
\sphinxstylestrong{/etc/hosts Entry}

\sphinxAtStartPar
Your laptop uses the following line in \sphinxcode{\sphinxupquote{/etc/hosts}}:

\begin{sphinxVerbatim}[commandchars=\\\{\}]
10.10.10.10 www.example.com
\end{sphinxVerbatim}

\sphinxAtStartPar
This resolves \sphinxcode{\sphinxupquote{www.example.com}} locally without DNS.

\sphinxAtStartPar
\sphinxstylestrong{Routing to the Cluster}
\begin{itemize}
\item {} 
\sphinxAtStartPar
The LXC router at \sphinxcode{\sphinxupquote{192.168.1.200}} forwards traffic to \sphinxcode{\sphinxupquote{10.10.10.10}}.

\item {} 
\sphinxAtStartPar
Routing may involve NAT or static rules to ensure reachability.

\end{itemize}


\section{5. How It All Ties Together}
\label{\detokenize{setup:how-it-all-ties-together}}
\sphinxAtStartPar
The full path of a request to \sphinxcode{\sphinxupquote{www.example.com}} is:
\begin{enumerate}
\sphinxsetlistlabels{\arabic}{enumi}{enumii}{}{.}%
\item {} 
\sphinxAtStartPar
\sphinxstylestrong{Laptop Resolution}
\sphinxhyphen{} The browser looks up \sphinxcode{\sphinxupquote{www.example.com}} and resolves it to \sphinxcode{\sphinxupquote{10.10.10.10}} using \sphinxcode{\sphinxupquote{/etc/hosts}}.

\item {} 
\sphinxAtStartPar
\sphinxstylestrong{Network Routing}
\sphinxhyphen{} Traffic is sent to \sphinxcode{\sphinxupquote{10.10.10.10}} and routed by the LXC container to the Talos node that owns this IP.

\item {} 
\sphinxAtStartPar
\sphinxstylestrong{MetalLB}
\sphinxhyphen{} Owns the \sphinxcode{\sphinxupquote{10.10.10.10}} IP and ensures the request reaches the correct node.

\item {} 
\sphinxAtStartPar
\sphinxstylestrong{Traefik}
\sphinxhyphen{} Receives the HTTP/S request on port 80/443.
\sphinxhyphen{} Uses the Ingress resource to determine routing.

\item {} 
\sphinxAtStartPar
\sphinxstylestrong{Kubernetes Service and Pods}
\sphinxhyphen{} \sphinxcode{\sphinxupquote{example\sphinxhyphen{}service}} forwards the request to one of your pods (e.g., Nginx).
\sphinxhyphen{} The response is returned back through Traefik and MetalLB.

\item {} 
\sphinxAtStartPar
\sphinxstylestrong{Response}
\sphinxhyphen{} Your browser displays the resulting web page from \sphinxcode{\sphinxupquote{www.example.com}}.

\end{enumerate}

\sphinxstepscope


\chapter{Talos}
\label{\detokenize{talos:talos}}\label{\detokenize{talos::doc}}
\sphinxincludegraphics[]{graphviz-dffe611f83d140a7da942840c23b4a3018ab3716.pdf}

\sphinxAtStartPar
Talos is a modern OS for Kubernetes. It is designed to be secure, immutable, and minimal. Talos is a self\sphinxhyphen{}hosted Kubernetes distribution that runs on bare metal or virtualized infrastructure. Talos is designed to be managed by a central Kubernetes control plane, which can be hosted on the same cluster or on a separate cluster.

\sphinxAtStartPar
talosctl config add my\sphinxhyphen{}cluster \textendash{}endpoints 192.168.0.242

\sphinxAtStartPar
talosctl config info

\sphinxAtStartPar
talosctl config endpoint 192.168.0.242

\sphinxAtStartPar
talosctl gen config my\sphinxhyphen{}cluster \sphinxurl{https://192.168.0.242:6443} \textendash{}output\sphinxhyphen{}dir ./talos\sphinxhyphen{}config \textendash{}force


\section{new install talos}
\label{\detokenize{talos:new-install-talos}}
\sphinxAtStartPar
\sphinxurl{https://www.talos.dev/v1.9/talos-guides/install/virtualized-platforms/proxmox/}
\begin{quote}
\begin{quote}

\sphinxAtStartPar
talosctl gen config my\sphinxhyphen{}cluster \sphinxurl{https://192.168.0.218:6443}
talosctl \sphinxhyphen{}n 192.168.0.169 get disks \textendash{}insecure (check disks)
talosctl config endpoint 192.168.0.218
talosctl config node 192.168.0.218

\sphinxAtStartPar
talosctl apply\sphinxhyphen{}config \textendash{}insecure \textendash{}nodes 192.168.0.218 \textendash{}file controlplane.yaml

\sphinxAtStartPar
talosctl bootstrap
talosctl kubeconfig . (retrieve kubeconfig)
talosctl \textendash{}nodes 192.168.0.218 version (verify)

\sphinxAtStartPar
export KUBECONFIG=./talos\sphinxhyphen{}config/kubeconfig
\begin{quote}

\sphinxAtStartPar
kubectl get nodes
kubectl get pods \sphinxhyphen{}n kube\sphinxhyphen{}system
kubectl get pods \sphinxhyphen{}n kube\sphinxhyphen{}system \sphinxhyphen{}o wide
\end{quote}
\end{quote}

\sphinxAtStartPar
kubectl describe pod my\sphinxhyphen{}postgres\sphinxhyphen{}postgresql\sphinxhyphen{}0 (is very useful in case the pod does get deployed

\sphinxAtStartPar
\sphinxurl{https://factory.talos.dev/} (create your custom image)

\begin{sphinxVerbatim}[commandchars=\\\{\}]
talosctl\PYG{+w}{ }upgrade\PYG{+w}{ }\PYGZhy{}\PYGZhy{}nodes\PYG{+w}{ }\PYG{l+m}{10}.10.10.178\PYG{+w}{ }\PYGZhy{}\PYGZhy{}image\PYG{+w}{  }factory.talos.dev/installer/c9078f9419961640c712a8bf2bb9174933dfcf1da383fd8ea2b7dc21493f8bac:v1.9.5
\end{sphinxVerbatim}
\end{quote}
\begin{description}
\sphinxlineitem{watching nodes: {[}10.10.10.178{]}}
\sphinxAtStartPar
talosctl get extensions \textendash{}nodes 10.10.10.178

\end{description}

\sphinxAtStartPar
NODE           NAMESPACE   TYPE              ID   VERSION   NAME          VERSION
10.10.10.178   runtime     ExtensionStatus   0    1         iscsi\sphinxhyphen{}tools   v0.1.6
10.10.10.178   runtime     ExtensionStatus   1    1         schematic     c9078f9419961640c712a8bf2bb9174933dfcf1da383fd8ea2b7dc21493f8bac


\section{adding worker nodes}
\label{\detokenize{talos:adding-worker-nodes}}
\sphinxAtStartPar
Since “longhorn” stores data on more than one node, we need to add more nodes to the cluster.
\begin{quote}

\sphinxAtStartPar
talosctl apply\sphinxhyphen{}config \textendash{}insecure \textendash{}nodes 10.10.10.166 \textendash{}file worker.yaml
talosctl apply\sphinxhyphen{}config \textendash{}insecure \textendash{}nodes 10.10.10.173 \textendash{}file worker.yaml
\end{quote}

\sphinxAtStartPar
kubectl get nodes \sphinxhyphen{}o wide
NAME            STATUS   ROLES           AGE     VERSION   INTERNAL\sphinxhyphen{}IP    EXTERNAL\sphinxhyphen{}IP   OS\sphinxhyphen{}IMAGE         KERNEL\sphinxhyphen{}VERSION   CONTAINER\sphinxhyphen{}RUNTIME
talos\sphinxhyphen{}2ho\sphinxhyphen{}roe   Ready    \textless{}none\textgreater{}          113s    v1.32.3   10.10.10.173   \textless{}none\textgreater{}        Talos (v1.9.5)   6.12.18\sphinxhyphen{}talos    containerd://2.0.3
talos\sphinxhyphen{}swn\sphinxhyphen{}isw   Ready    control\sphinxhyphen{}plane   31d     v1.32.3   10.10.10.118   \textless{}none\textgreater{}        Talos (v1.9.5)   6.12.18\sphinxhyphen{}talos    containerd://2.0.3
talos\sphinxhyphen{}v1x\sphinxhyphen{}9s4   Ready    \textless{}none\textgreater{}          2m18s   v1.32.3   10.10.10.166   \textless{}none\textgreater{}        Talos (v1.9.5)   6.12.18\sphinxhyphen{}talos    containerd://2.0.3
talos\sphinxhyphen{}y7t\sphinxhyphen{}8ll   Ready    worker          29d     v1.32.3   10.10.10.178   \textless{}none\textgreater{}        Talos (v1.9.5)   6.12.18\sphinxhyphen{}talos    containerd://2.0.3


\section{label nodes}
\label{\detokenize{talos:label-nodes}}\begin{quote}

\sphinxAtStartPar
kubectl label nodes talos\sphinxhyphen{}v1x\sphinxhyphen{}9s4 node\sphinxhyphen{}role.kubernetes.io/worker=””
kubectl label nodes talos\sphinxhyphen{}2ho\sphinxhyphen{}roe node\sphinxhyphen{}role.kubernetes.io/worker=””
\end{quote}

\sphinxstepscope


\chapter{Proxmox  installation notes}
\label{\detokenize{cluster:proxmox-installation-notes}}\label{\detokenize{cluster::doc}}
\sphinxAtStartPar
remove firewall from talos worker node

\begin{sphinxVerbatim}[commandchars=\\\{\}]
root@pve:\PYGZti{}\PYGZsh{}\PYG{+w}{ }qm\PYG{+w}{ }\PYG{n+nb}{set}\PYG{+w}{ }\PYG{l+m}{109}\PYG{+w}{ }\PYGZhy{}net0\PYG{+w}{ }virtio,bridge\PYG{o}{=}vmbr0,firewall\PYG{o}{=}\PYG{l+m}{0}
update\PYG{+w}{ }VM\PYG{+w}{ }\PYG{l+m}{109}:\PYG{+w}{ }\PYGZhy{}net0\PYG{+w}{ }virtio,bridge\PYG{o}{=}vmbr0,firewall\PYG{o}{=}\PYG{l+m}{0}
\end{sphinxVerbatim}


\section{setting up an extra network bridge vmbr1}
\label{\detokenize{cluster:setting-up-an-extra-network-bridge-vmbr1}}
\sphinxAtStartPar
on lxc dedicated machine setup dhcp and routing

\begin{sphinxVerbatim}[commandchars=\\\{\}]
apt\PYG{+w}{ }install\PYG{+w}{ }dnsmasq\PYG{+w}{ }\PYGZhy{}y
apt\PYG{+w}{ }install\PYG{+w}{ }iptables\PYGZhy{}persistent\PYG{+w}{ }\PYGZhy{}y

vi\PYG{+w}{ }/etc/dnsmasq.conf
\PYG{n+nv}{interface}\PYG{o}{=}eth0
dhcp\PYGZhy{}range\PYG{o}{=}\PYG{l+m}{10}.10.10.100,10.10.10.200,12h\PYG{+w}{  }\PYG{c+c1}{\PYGZsh{} DHCP range for Talos nodes}
dhcp\PYGZhy{}option\PYG{o}{=}\PYG{l+m}{3},10.10.10.2\PYG{+w}{                  }\PYG{c+c1}{\PYGZsh{} Gateway (this machine’s eth0 IP)}
dhcp\PYGZhy{}option\PYG{o}{=}\PYG{l+m}{6},192.168.0.1\PYG{+w}{                 }\PYG{c+c1}{\PYGZsh{} DNS (your home router’s DNS)}

systemctl\PYG{+w}{ }restart\PYG{+w}{ }dnsmasq

\PYG{n+nb}{echo}\PYG{+w}{ }\PYG{l+m}{1}\PYG{+w}{ }\PYGZgt{}\PYG{+w}{ }/proc/sys/net/ipv4/ip\PYGZus{}forward

iptables\PYG{+w}{ }\PYGZhy{}t\PYG{+w}{ }nat\PYG{+w}{ }\PYGZhy{}L\PYG{+w}{ }\PYGZhy{}v
ip\PYG{+w}{ }route\PYG{+w}{ }del\PYG{+w}{ }default\PYG{+w}{ }via\PYG{+w}{ }\PYG{l+m}{10}.10.10.1\PYG{+w}{ }dev\PYG{+w}{ }eth0
ip\PYG{+w}{ }route\PYG{+w}{ }replace\PYG{+w}{ }default\PYG{+w}{ }via\PYG{+w}{ }\PYG{l+m}{192}.168.0.1\PYG{+w}{ }dev\PYG{+w}{ }eth1\PYG{+w}{ }metric\PYG{+w}{ }\PYG{l+m}{100}

iptables\PYG{+w}{ }\PYGZhy{}t\PYG{+w}{ }nat\PYG{+w}{ }\PYGZhy{}A\PYG{+w}{ }POSTROUTING\PYG{+w}{ }\PYGZhy{}s\PYG{+w}{ }\PYG{l+m}{10}.10.10.0/24\PYG{+w}{ }\PYGZhy{}o\PYG{+w}{ }eth1\PYG{+w}{ }\PYGZhy{}j\PYG{+w}{ }MASQUERADE
ip\PYG{+w}{ }route\PYG{+w}{ }del\PYG{+w}{ }default\PYG{+w}{ }via\PYG{+w}{ }\PYG{l+m}{10}.10.10.1\PYG{+w}{ }dev\PYG{+w}{ }eth0
ip\PYG{+w}{ }route\PYG{+w}{ }replace\PYG{+w}{ }default\PYG{+w}{ }via\PYG{+w}{ }\PYG{l+m}{192}.168.0.1\PYG{+w}{ }dev\PYG{+w}{ }eth1\PYG{+w}{ }metric\PYG{+w}{ }\PYG{l+m}{100}



vi\PYG{+w}{ }/etc/netplan/01\PYGZhy{}netcfg.yaml
\end{sphinxVerbatim}

\begin{sphinxVerbatim}[commandchars=\\\{\}]
\PYG{n+nt}{network}\PYG{p}{:}
\PYG{+w}{  }\PYG{n+nt}{version}\PYG{p}{:}\PYG{+w}{ }\PYG{l+lScalar+lScalarPlain}{2}
\PYG{+w}{  }\PYG{n+nt}{ethernets}\PYG{p}{:}
\PYG{+w}{    }\PYG{n+nt}{eth0}\PYG{p}{:}
\PYG{+w}{      }\PYG{n+nt}{dhcp4}\PYG{p}{:}\PYG{+w}{ }\PYG{l+lScalar+lScalarPlain}{true}
\PYG{+w}{      }\PYG{c+c1}{\PYGZsh{} Prevent DHCP from setting a default gateway if it conflicts}
\PYG{+w}{      }\PYG{n+nt}{dhcp4\PYGZhy{}overrides}\PYG{p}{:}
\PYG{+w}{        }\PYG{n+nt}{use\PYGZhy{}routes}\PYG{p}{:}\PYG{+w}{ }\PYG{l+lScalar+lScalarPlain}{true}
\PYG{+w}{        }\PYG{n+nt}{use\PYGZhy{}dns}\PYG{p}{:}\PYG{+w}{ }\PYG{l+lScalar+lScalarPlain}{true}
\PYG{+w}{        }\PYG{n+nt}{route\PYGZhy{}metric}\PYG{p}{:}\PYG{+w}{ }\PYG{l+lScalar+lScalarPlain}{2000}\PYG{+w}{  }\PYG{c+c1}{\PYGZsh{} High metric to prioritize eth1\PYGZsq{}s default route}
\PYG{+w}{    }\PYG{n+nt}{eth1}\PYG{p}{:}
\PYG{+w}{      }\PYG{n+nt}{dhcp4}\PYG{p}{:}\PYG{+w}{ }\PYG{l+lScalar+lScalarPlain}{false}
\PYG{+w}{      }\PYG{n+nt}{addresses}\PYG{p}{:}
\PYG{+w}{        }\PYG{p+pIndicator}{\PYGZhy{}}\PYG{+w}{ }\PYG{l+lScalar+lScalarPlain}{192.168.0.x/24}\PYG{+w}{  }\PYG{c+c1}{\PYGZsh{} Replace with your server\PYGZsq{}s IP on this subnet}
\PYG{+w}{      }\PYG{n+nt}{routes}\PYG{p}{:}
\PYG{+w}{        }\PYG{p+pIndicator}{\PYGZhy{}}\PYG{+w}{ }\PYG{n+nt}{to}\PYG{p}{:}\PYG{+w}{ }\PYG{l+lScalar+lScalarPlain}{0.0.0.0/0}
\PYG{+w}{          }\PYG{n+nt}{via}\PYG{p}{:}\PYG{+w}{ }\PYG{l+lScalar+lScalarPlain}{192.168.0.1}
\PYG{+w}{          }\PYG{n+nt}{metric}\PYG{p}{:}\PYG{+w}{ }\PYG{l+lScalar+lScalarPlain}{100}
\PYG{+w}{        }\PYG{p+pIndicator}{\PYGZhy{}}\PYG{+w}{ }\PYG{n+nt}{to}\PYG{p}{:}\PYG{+w}{ }\PYG{l+lScalar+lScalarPlain}{0.0.0.0/0}
\PYG{+w}{          }\PYG{n+nt}{via}\PYG{p}{:}\PYG{+w}{ }\PYG{l+lScalar+lScalarPlain}{192.168.0.1}
\PYG{+w}{          }\PYG{n+nt}{metric}\PYG{p}{:}\PYG{+w}{ }\PYG{l+lScalar+lScalarPlain}{1024}
\PYG{+w}{        }\PYG{p+pIndicator}{\PYGZhy{}}\PYG{+w}{ }\PYG{n+nt}{to}\PYG{p}{:}\PYG{+w}{ }\PYG{l+lScalar+lScalarPlain}{192.168.0.0/24}
\PYG{+w}{          }\PYG{n+nt}{via}\PYG{p}{:}\PYG{+w}{ }\PYG{l+lScalar+lScalarPlain}{0.0.0.0}
\PYG{+w}{          }\PYG{n+nt}{metric}\PYG{p}{:}\PYG{+w}{ }\PYG{l+lScalar+lScalarPlain}{1024}
\PYG{+w}{        }\PYG{p+pIndicator}{\PYGZhy{}}\PYG{+w}{ }\PYG{n+nt}{to}\PYG{p}{:}\PYG{+w}{ }\PYG{l+lScalar+lScalarPlain}{192.168.0.1}
\PYG{+w}{          }\PYG{n+nt}{via}\PYG{p}{:}\PYG{+w}{ }\PYG{l+lScalar+lScalarPlain}{0.0.0.0}
\PYG{+w}{          }\PYG{n+nt}{metric}\PYG{p}{:}\PYG{+w}{ }\PYG{l+lScalar+lScalarPlain}{1024}
\PYG{+w}{        }\PYG{p+pIndicator}{\PYGZhy{}}\PYG{+w}{ }\PYG{n+nt}{to}\PYG{p}{:}\PYG{+w}{ }\PYG{l+lScalar+lScalarPlain}{\PYGZlt{}gent.dnscache01\PYGZhy{}ip\PYGZgt{}}
\PYG{+w}{          }\PYG{n+nt}{via}\PYG{p}{:}\PYG{+w}{ }\PYG{l+lScalar+lScalarPlain}{192.168.0.1}
\PYG{+w}{          }\PYG{n+nt}{metric}\PYG{p}{:}\PYG{+w}{ }\PYG{l+lScalar+lScalarPlain}{1024}
\PYG{+w}{        }\PYG{p+pIndicator}{\PYGZhy{}}\PYG{+w}{ }\PYG{n+nt}{to}\PYG{p}{:}\PYG{+w}{ }\PYG{l+lScalar+lScalarPlain}{\PYGZlt{}gent.dnscache02\PYGZhy{}ip\PYGZgt{}}
\PYG{+w}{          }\PYG{n+nt}{via}\PYG{p}{:}\PYG{+w}{ }\PYG{l+lScalar+lScalarPlain}{192.168.0.1}
\PYG{+w}{          }\PYG{n+nt}{metric}\PYG{p}{:}\PYG{+w}{ }\PYG{l+lScalar+lScalarPlain}{1024}
\end{sphinxVerbatim}

\begin{sphinxVerbatim}[commandchars=\\\{\}]
\PYG{c+c1}{\PYGZsh{} Apply the netplan configuration}
sudo\PYG{+w}{ }netplan\PYG{+w}{ }generate
sudo\PYG{+w}{ }netplan\PYG{+w}{ }apply

\PYG{c+c1}{\PYGZsh{} Check the routing table}
ip\PYG{+w}{ }route\PYG{+w}{ }show

\PYG{c+c1}{\PYGZsh{} Check iptables rules}
iptables\PYG{+w}{ }\PYGZhy{}t\PYG{+w}{ }nat\PYG{+w}{ }\PYGZhy{}L\PYG{+w}{ }\PYGZhy{}v

\PYG{c+c1}{\PYGZsh{} Check dnsmasq status}
systemctl\PYG{+w}{ }status\PYG{+w}{ }dnsmasq

\PYG{c+c1}{\PYGZsh{} Check if the DHCP server is running and listening on the correct interface}
sudo\PYG{+w}{ }systemctl\PYG{+w}{ }status\PYG{+w}{ }dnsmasq

\PYG{c+c1}{\PYGZsh{} Restart dnsmasq to apply changes}
sudo\PYG{+w}{ }systemctl\PYG{+w}{ }restart\PYG{+w}{ }dnsmasq


netplan\PYG{+w}{ }apply
\end{sphinxVerbatim}


\section{Kernel IP routing table}
\label{\detokenize{cluster:kernel-ip-routing-table}}

\section{using the nodeport}
\label{\detokenize{cluster:using-the-nodeport}}
\sphinxAtStartPar
192.168.0.251:30743

\sphinxAtStartPar
on my router/dhcp on 10.10.10.2 route port to cluster node IP

\sphinxAtStartPar
iptables \sphinxhyphen{}t nat \sphinxhyphen{}A PREROUTING \sphinxhyphen{}i eth1 \sphinxhyphen{}p tcp \textendash{}dport 30743 \sphinxhyphen{}j DNAT \textendash{}to\sphinxhyphen{}destination 10.10.10.118:30743

\sphinxAtStartPar
so running nginx on kubernetes on 10.10.10.255 network is accessible from the outside


\section{using the IP address}
\label{\detokenize{cluster:using-the-ip-address}}
\sphinxAtStartPar
traefik      LoadBalancer   10.102.122.212   10.10.10.50   80:32178/TCP,443:32318/TCP   75m   app.kubernetes.io/instance=traefik\sphinxhyphen{}default,app.kubernetes.io/name=traefik

\sphinxAtStartPar
So now I have to figure out how I can reach  10.10.10.50 from my 192.168.X.X network

\sphinxAtStartPar
on the kubernetes cluster, traefik has been deployed as well as metallb.
iptables \sphinxhyphen{}t nat \sphinxhyphen{}A POSTROUTING \sphinxhyphen{}s 192.168.0.0/24 \sphinxhyphen{}d 10.10.10.0/24 \sphinxhyphen{}j MASQUERADE
sh \sphinxhyphen{}c “iptables\sphinxhyphen{}save \textgreater{} /etc/iptables/rules.v4”

\sphinxAtStartPar
this has been added to th dnsmasq.conf

\sphinxAtStartPar
\# Listen on the 192.168.0.251 interface
interface=eth1  \# Replace with your 192.168.0.251 interface (check with \sphinxtitleref{ip a})
listen\sphinxhyphen{}address=192.168.0.251

\sphinxAtStartPar
\# Forward other queries to upstream DNS (e.g., Google DNS)
server=8.8.8.8
server=8.8.4.4

\sphinxAtStartPar
\# Optional: If LXC is your DHCP server, ensure DNS is offered
dhcp\sphinxhyphen{}option=6,192.168.0.251  \# Tells DHCP clients to use this as DNS


\section{modify dns config on laptop}
\label{\detokenize{cluster:modify-dns-config-on-laptop}}
\sphinxAtStartPar
/etc/resolv.conf

\sphinxAtStartPar
add : nameserver 192.168.0.251


\section{access http://nginx.example.com/ on talos within 10.10.10.X from 192.168.X.X}
\label{\detokenize{cluster:access-http-nginx-example-com-on-talos-within-10-10-10-x-from-192-168-x-x}}
\sphinxAtStartPar
(configure metallb, traefik, nginx)

\sphinxAtStartPar
on laptop /etc/hosts : 10.10.10.50 nginx.example.com

\sphinxAtStartPar
on dhcp server (10.10.10.2)

\sphinxAtStartPar
iptables \sphinxhyphen{}A FORWARD \sphinxhyphen{}s 192.168.0.0/24 \sphinxhyphen{}d 10.10.10.0/24 \sphinxhyphen{}j ACCEPT
iptables \sphinxhyphen{}A FORWARD \sphinxhyphen{}s 10.10.10.0/24 \sphinxhyphen{}d 192.168.0.0/24 \sphinxhyphen{}j ACCEPT

\begin{sphinxVerbatim}[commandchars=\\\{\}]
\PYG{c+c1}{\PYGZsh{} Generated by iptables\PYGZhy{}save v1.8.7 on Thu Apr 10 13:32:59 2025}
*filter
:INPUT\PYG{+w}{ }ACCEPT\PYG{+w}{ }\PYG{o}{[}\PYG{l+m}{0}:0\PYG{o}{]}
:FORWARD\PYG{+w}{ }ACCEPT\PYG{+w}{ }\PYG{o}{[}\PYG{l+m}{0}:0\PYG{o}{]}
:OUTPUT\PYG{+w}{ }ACCEPT\PYG{+w}{ }\PYG{o}{[}\PYG{l+m}{0}:0\PYG{o}{]}
\PYGZhy{}A\PYG{+w}{ }FORWARD\PYG{+w}{ }\PYGZhy{}s\PYG{+w}{ }\PYG{l+m}{192}.168.0.0/24\PYG{+w}{ }\PYGZhy{}d\PYG{+w}{ }\PYG{l+m}{10}.10.10.0/24\PYG{+w}{ }\PYGZhy{}j\PYG{+w}{ }ACCEPT
\PYGZhy{}A\PYG{+w}{ }FORWARD\PYG{+w}{ }\PYGZhy{}s\PYG{+w}{ }\PYG{l+m}{10}.10.10.0/24\PYG{+w}{ }\PYGZhy{}d\PYG{+w}{ }\PYG{l+m}{192}.168.0.0/24\PYG{+w}{ }\PYGZhy{}j\PYG{+w}{ }ACCEPT
COMMIT
\PYG{c+c1}{\PYGZsh{} Completed on Thu Apr 10 13:32:59 2025}
\PYG{c+c1}{\PYGZsh{} Generated by iptables\PYGZhy{}save v1.8.7 on Thu Apr 10 13:32:59 2025}
*nat
:PREROUTING\PYG{+w}{ }ACCEPT\PYG{+w}{ }\PYG{o}{[}\PYG{l+m}{6847}:1975161\PYG{o}{]}
:INPUT\PYG{+w}{ }ACCEPT\PYG{+w}{ }\PYG{o}{[}\PYG{l+m}{158}:15156\PYG{o}{]}
:OUTPUT\PYG{+w}{ }ACCEPT\PYG{+w}{ }\PYG{o}{[}\PYG{l+m}{25}:2590\PYG{o}{]}
:POSTROUTING\PYG{+w}{ }ACCEPT\PYG{+w}{ }\PYG{o}{[}\PYG{l+m}{25}:2590\PYG{o}{]}
\PYGZhy{}A\PYG{+w}{ }PREROUTING\PYG{+w}{ }\PYGZhy{}i\PYG{+w}{ }eth1\PYG{+w}{ }\PYGZhy{}p\PYG{+w}{ }tcp\PYG{+w}{ }\PYGZhy{}m\PYG{+w}{ }tcp\PYG{+w}{ }\PYGZhy{}\PYGZhy{}dport\PYG{+w}{ }\PYG{l+m}{30743}\PYG{+w}{ }\PYGZhy{}j\PYG{+w}{ }DNAT\PYG{+w}{ }\PYGZhy{}\PYGZhy{}to\PYGZhy{}destination\PYG{+w}{ }\PYG{l+m}{10}.10.10.118:30743
\PYGZhy{}A\PYG{+w}{ }POSTROUTING\PYG{+w}{ }\PYGZhy{}s\PYG{+w}{ }\PYG{l+m}{10}.10.10.0/24\PYG{+w}{ }\PYGZhy{}o\PYG{+w}{ }eth1\PYG{+w}{ }\PYGZhy{}j\PYG{+w}{ }MASQUERADE
\PYGZhy{}A\PYG{+w}{ }POSTROUTING\PYG{+w}{ }\PYGZhy{}s\PYG{+w}{ }\PYG{l+m}{10}.10.10.0/24\PYG{+w}{ }\PYGZhy{}o\PYG{+w}{ }eth1\PYG{+w}{ }\PYGZhy{}j\PYG{+w}{ }MASQUERADE
\PYGZhy{}A\PYG{+w}{ }POSTROUTING\PYG{+w}{ }\PYGZhy{}s\PYG{+w}{ }\PYG{l+m}{10}.10.10.0/24\PYG{+w}{ }\PYGZhy{}o\PYG{+w}{ }eth1\PYG{+w}{ }\PYGZhy{}j\PYG{+w}{ }MASQUERADE
\PYGZhy{}A\PYG{+w}{ }POSTROUTING\PYG{+w}{ }\PYGZhy{}s\PYG{+w}{ }\PYG{l+m}{192}.168.0.0/24\PYG{+w}{ }\PYGZhy{}d\PYG{+w}{ }\PYG{l+m}{10}.10.10.0/24\PYG{+w}{ }\PYGZhy{}j\PYG{+w}{ }MASQUERADE
COMMIT
\PYG{c+c1}{\PYGZsh{} Completed on Thu Apr 10 13:32:59 2025}
\PYG{c+c1}{\PYGZsh{} Check the iptables rules}
iptables\PYG{+w}{ }\PYGZhy{}t\PYG{+w}{ }nat\PYG{+w}{ }\PYGZhy{}L\PYG{+w}{ }\PYGZhy{}v
iptables\PYG{+w}{ }\PYGZhy{}L\PYG{+w}{ }\PYGZhy{}v

\PYG{c+c1}{\PYGZsh{} Check the routing table}
ip\PYG{+w}{ }route\PYG{+w}{ }show

\PYG{c+c1}{\PYGZsh{} Check the network interfaces}
ip\PYG{+w}{ }a
\end{sphinxVerbatim}

\sphinxstepscope


\chapter{Kernel IP Routing Table}
\label{\detokenize{routing-table:kernel-ip-routing-table}}\label{\detokenize{routing-table::doc}}
\sphinxAtStartPar
The following table represents the kernel IP routing table:


\begin{savenotes}\sphinxattablestart
\sphinxthistablewithglobalstyle
\centering
\sphinxcapstartof{table}
\sphinxthecaptionisattop
\sphinxcaption{Kernel IP Routing Table}\label{\detokenize{routing-table:id1}}
\sphinxaftertopcaption
\begin{tabular}[t]{\X{15}{75}\X{15}{75}\X{15}{75}\X{5}{75}\X{5}{75}\X{5}{75}\X{5}{75}\X{10}{75}}
\sphinxtoprule
\sphinxstyletheadfamily 
\sphinxAtStartPar
Destination
&\sphinxstyletheadfamily 
\sphinxAtStartPar
Gateway
&\sphinxstyletheadfamily 
\sphinxAtStartPar
Genmask
&\sphinxstyletheadfamily 
\sphinxAtStartPar
Flags
&\sphinxstyletheadfamily 
\sphinxAtStartPar
Metric
&\sphinxstyletheadfamily 
\sphinxAtStartPar
Ref
&\sphinxstyletheadfamily 
\sphinxAtStartPar
Use
&\sphinxstyletheadfamily 
\sphinxAtStartPar
Iface
\\
\sphinxmidrule
\sphinxtableatstartofbodyhook
\sphinxAtStartPar
default
&
\sphinxAtStartPar
192.168.0.1
&
\sphinxAtStartPar
0.0.0.0
&
\sphinxAtStartPar
UG
&
\sphinxAtStartPar
100
&
\sphinxAtStartPar
0
&
\sphinxAtStartPar
0
&
\sphinxAtStartPar
eth1
\\
\sphinxhline
\sphinxAtStartPar
default
&
\sphinxAtStartPar
192.168.0.1
&
\sphinxAtStartPar
0.0.0.0
&
\sphinxAtStartPar
UG
&
\sphinxAtStartPar
1024
&
\sphinxAtStartPar
0
&
\sphinxAtStartPar
0
&
\sphinxAtStartPar
eth1
\\
\sphinxhline
\sphinxAtStartPar
10.10.10.0
&
\sphinxAtStartPar
0.0.0.0
&
\sphinxAtStartPar
255.255.255.0
&
\sphinxAtStartPar
U
&
\sphinxAtStartPar
0
&
\sphinxAtStartPar
0
&
\sphinxAtStartPar
0
&
\sphinxAtStartPar
eth0
\\
\sphinxhline
\sphinxAtStartPar
192.168.0.0
&
\sphinxAtStartPar
0.0.0.0
&
\sphinxAtStartPar
255.255.255.0
&
\sphinxAtStartPar
U
&
\sphinxAtStartPar
1024
&
\sphinxAtStartPar
0
&
\sphinxAtStartPar
0
&
\sphinxAtStartPar
eth1
\\
\sphinxhline
\sphinxAtStartPar
192.168.0.1
&
\sphinxAtStartPar
0.0.0.0
&
\sphinxAtStartPar
255.255.255.255
&
\sphinxAtStartPar
UH
&
\sphinxAtStartPar
1024
&
\sphinxAtStartPar
0
&
\sphinxAtStartPar
0
&
\sphinxAtStartPar
eth1
\\
\sphinxhline
\sphinxAtStartPar
gent.dnscache01
&
\sphinxAtStartPar
192.168.0.1
&
\sphinxAtStartPar
255.255.255.255
&
\sphinxAtStartPar
UGH
&
\sphinxAtStartPar
1024
&
\sphinxAtStartPar
0
&
\sphinxAtStartPar
0
&
\sphinxAtStartPar
eth1
\\
\sphinxhline
\sphinxAtStartPar
gent.dnscache02
&
\sphinxAtStartPar
192.168.0.1
&
\sphinxAtStartPar
255.255.255.255
&
\sphinxAtStartPar
UGH
&
\sphinxAtStartPar
1024
&
\sphinxAtStartPar
0
&
\sphinxAtStartPar
0
&
\sphinxAtStartPar
eth1
\\
\sphinxbottomrule
\end{tabular}
\sphinxtableafterendhook\par
\sphinxattableend\end{savenotes}

\sphinxstepscope


\chapter{Laptop (or PC) mods}
\label{\detokenize{laptop:laptop-or-pc-mods}}\label{\detokenize{laptop::doc}}
\sphinxAtStartPar
/etc/hosts

\sphinxAtStartPar
10.10.10.50 nginx.example.com
10.10.10.50 registry.example.com
\begin{quote}

\sphinxAtStartPar
ip route add 10.10.10.0/24 via 192.168.0.251
\end{quote}

\sphinxAtStartPar
sudo cp ca.crt /usr/local/share/ca\sphinxhyphen{}certificates/ca.crt
sudo update\sphinxhyphen{}ca\sphinxhyphen{}certificates
Updating certificates in /etc/ssl/certs…


\section{make route permanent}
\label{\detokenize{laptop:make-route-permanent}}
\begin{sphinxVerbatim}[commandchars=\\\{\}]
\PYG{l+lScalar+lScalarPlain}{sudo}\PYG{l+lScalar+lScalarPlain}{ }\PYG{l+lScalar+lScalarPlain}{nano}\PYG{l+lScalar+lScalarPlain}{ }\PYG{l+lScalar+lScalarPlain}{/etc/netplan/01\PYGZhy{}netcfg.yaml}

\PYG{l+lScalar+lScalarPlain}{network}\PYG{p+pIndicator}{:}
\PYG{n+nt}{version}\PYG{p}{:}\PYG{+w}{ }\PYG{l+lScalar+lScalarPlain}{2}
\PYG{n+nt}{ethernets}\PYG{p}{:}
\PYG{+w}{  }\PYG{n+nt}{wlp0s20f3}\PYG{p}{:}
\PYG{+w}{    }\PYG{n+nt}{addresses}\PYG{p}{:}
\PYG{+w}{      }\PYG{p+pIndicator}{\PYGZhy{}}\PYG{+w}{ }\PYG{l+lScalar+lScalarPlain}{192.168.0.103/24}
\PYG{+w}{    }\PYG{n+nt}{gateway4}\PYG{p}{:}\PYG{+w}{ }\PYG{l+lScalar+lScalarPlain}{192.168.0.1}\PYG{+w}{  }\PYG{c+c1}{\PYGZsh{} Replace with your actual default gateway}
\PYG{+w}{    }\PYG{n+nt}{routes}\PYG{p}{:}
\PYG{+w}{      }\PYG{p+pIndicator}{\PYGZhy{}}\PYG{+w}{ }\PYG{n+nt}{to}\PYG{p}{:}\PYG{+w}{ }\PYG{l+lScalar+lScalarPlain}{10.10.10.0/24}
\PYG{+w}{        }\PYG{n+nt}{via}\PYG{p}{:}\PYG{+w}{ }\PYG{l+lScalar+lScalarPlain}{192.168.0.251}

\PYG{+w}{  }\PYG{l+lScalar+lScalarPlain}{sudo}\PYG{l+lScalar+lScalarPlain}{ }\PYG{l+lScalar+lScalarPlain}{netplan}\PYG{l+lScalar+lScalarPlain}{ }\PYG{l+lScalar+lScalarPlain}{apply}
\end{sphinxVerbatim}

\sphinxstepscope


\chapter{SDA (software defined architecture)}
\label{\detokenize{SDA:sda-software-defined-architecture}}\label{\detokenize{SDA::doc}}

\section{Physical Infrastructure}
\label{\detokenize{SDA:physical-infrastructure}}

\section{Proxmox Server Specifications}
\label{\detokenize{SDA:proxmox-server-specifications}}
\sphinxAtStartPar
The foundation of the architecture is a physical server running Proxmox VE hypervisor.


\begin{savenotes}\sphinxattablestart
\sphinxthistablewithglobalstyle
\centering
\begin{tabular}[t]{\X{30}{100}\X{70}{100}}
\sphinxtoprule
\sphinxstyletheadfamily 
\sphinxAtStartPar
Component
&\sphinxstyletheadfamily 
\sphinxAtStartPar
Description
\\
\sphinxmidrule
\sphinxtableatstartofbodyhook
\sphinxAtStartPar
\sphinxstylestrong{Hypervisor}
&
\sphinxAtStartPar
Proxmox Virtual Environment (VE)
\\
\sphinxhline
\sphinxAtStartPar
\sphinxstylestrong{Virtual Machines}
&
\sphinxAtStartPar
Multiple Talos OS nodes forming a Kubernetes cluster
\\
\sphinxhline
\sphinxAtStartPar
\sphinxstylestrong{Containers}
&
\sphinxAtStartPar
LXC container serving as a router
\\
\sphinxbottomrule
\end{tabular}
\sphinxtableafterendhook\par
\sphinxattableend\end{savenotes}


\section{Virtual Machine Configuration}
\label{\detokenize{SDA:virtual-machine-configuration}}
\begin{sphinxadmonition}{note}{Talos OS Nodes}

\sphinxAtStartPar
The cluster consists of multiple Talos OS nodes, with dedicated roles:
\begin{itemize}
\item {} 
\sphinxAtStartPar
\sphinxstylestrong{Control Plane Node(s)}: Manages the Kubernetes control plane

\item {} 
\sphinxAtStartPar
\sphinxstylestrong{Worker Nodes}: Runs application workloads

\end{itemize}
\end{sphinxadmonition}

\begin{sphinxVerbatim}[commandchars=\\\{\}]
\PYG{g+gp}{\PYGZsh{} }Example\PYG{+w}{ }Talos\PYG{+w}{ }configuration\PYG{+w}{ }structure\PYG{+w}{ }\PYG{o}{(}simplified\PYG{o}{)}
\PYG{g+go}{machine:}
\PYG{g+go}{  type: controlplane  \PYGZsh{} or worker}
\PYG{g+go}{  network:}
\PYG{g+go}{    hostname: talos\PYGZhy{}node\PYGZhy{}1}
\PYG{g+go}{  kubernetes:}
\PYG{g+go}{    version: v1.26.0}
\end{sphinxVerbatim}


\section{Networking Architecture}
\label{\detokenize{SDA:networking-architecture}}

\subsection{Network Components}
\label{\detokenize{SDA:network-components}}

\begin{savenotes}\sphinxattablestart
\sphinxthistablewithglobalstyle
\centering
\begin{tabular}[t]{\X{30}{100}\X{70}{100}}
\sphinxtoprule
\sphinxstyletheadfamily 
\sphinxAtStartPar
Component
&\sphinxstyletheadfamily 
\sphinxAtStartPar
Function
\\
\sphinxmidrule
\sphinxtableatstartofbodyhook
\sphinxAtStartPar
\sphinxstylestrong{Proxmox Virtual Bridge}
&
\sphinxAtStartPar
Creates isolated network segments for VMs and containers
\\
\sphinxhline
\sphinxAtStartPar
\sphinxstylestrong{LXC Router}
&
\sphinxAtStartPar
Routes traffic between internal and external networks
\\
\sphinxhline
\sphinxAtStartPar
\sphinxstylestrong{Kubernetes Overlay Network}
&
\sphinxAtStartPar
Enables pod\sphinxhyphen{}to\sphinxhyphen{}pod communication (Cilium, Flannel, etc.)
\\
\sphinxbottomrule
\end{tabular}
\sphinxtableafterendhook\par
\sphinxattableend\end{savenotes}


\section{Control \& Automation}
\label{\detokenize{SDA:control-automation}}

\subsection{API Management Layer}
\label{\detokenize{SDA:api-management-layer}}
\sphinxAtStartPar
This architecture leverages multiple declarative APIs for infrastructure management:


\begin{savenotes}\sphinxattablestart
\sphinxthistablewithglobalstyle
\centering
\begin{tabular}[t]{\X{25}{100}\X{75}{100}}
\sphinxtoprule
\sphinxstyletheadfamily 
\sphinxAtStartPar
API
&\sphinxstyletheadfamily 
\sphinxAtStartPar
Responsibility
\\
\sphinxmidrule
\sphinxtableatstartofbodyhook
\sphinxAtStartPar
\sphinxstylestrong{Proxmox API}
&
\sphinxAtStartPar
Manages physical resources, VMs, and containers
\\
\sphinxhline
\sphinxAtStartPar
\sphinxstylestrong{Talos API}
&
\sphinxAtStartPar
Provides declarative OS configuration and maintenance
\\
\sphinxhline
\sphinxAtStartPar
\sphinxstylestrong{Kubernetes API}
&
\sphinxAtStartPar
Orchestrates applications and services
\\
\sphinxbottomrule
\end{tabular}
\sphinxtableafterendhook\par
\sphinxattableend\end{savenotes}


\section{Benefits of This Architecture}
\label{\detokenize{SDA:benefits-of-this-architecture}}\begin{itemize}
\item {} 
\sphinxAtStartPar
\sphinxstylestrong{Immutable Infrastructure}: Talos OS provides an immutable, declarative operating system

\item {} 
\sphinxAtStartPar
\sphinxstylestrong{High Availability}: Kubernetes manages service availability and distribution

\item {} 
\sphinxAtStartPar
\sphinxstylestrong{Resource Efficiency}: Consolidates multiple services on a single physical server

\item {} 
\sphinxAtStartPar
\sphinxstylestrong{Isolation}: Separate network segments and container boundaries

\item {} 
\sphinxAtStartPar
\sphinxstylestrong{Automation}: API\sphinxhyphen{}driven management at all levels

\end{itemize}


\chapter{Indices and tables}
\label{\detokenize{index:indices-and-tables}}\begin{itemize}
\item {} 
\sphinxAtStartPar
\DUrole{xref,std,std-ref}{genindex}

\item {} 
\sphinxAtStartPar
\DUrole{xref,std,std-ref}{modindex}

\item {} 
\sphinxAtStartPar
\DUrole{xref,std,std-ref}{search}

\end{itemize}



\renewcommand{\indexname}{Index}
\printindex
\end{document}